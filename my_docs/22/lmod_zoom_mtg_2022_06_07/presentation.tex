\documentclass{beamer}

% You can also use a 16:9 aspect ratio:
%\documentclass[aspectratio=169]{beamer}
\usetheme{TACC16}

% It's possible to move the footer to the right:
%\usetheme[rightfooter]{TACC16}

%% page 
%\begin{frame}{}
%  \begin{itemize}
%    \item
%  \end{itemize}
%\end{frame}
%
%% page 
%\begin{frame}[fragile]
%    \frametitle{}
% {\tiny
%    \begin{semiverbatim}
%    \end{semiverbatim}
%}
%  \begin{itemize}
%    \item
%  \end{itemize}
%
%\end{frame}

\begin{document}
\title[Lmod]{How Lmod Loads a Modulefile, Part 1}
\author{Robert McLay} 
\date{June. 7, 2022}

% page 1
\frame{\titlepage} 


% page 2
\begin{frame}{Outline}
  \center{\includegraphics[width=.9\textwidth]{Lmod-4color@2x.png}}
  \begin{itemize}
    \item Talk: How Lmod finds a modulefile to load
    \item Next Time: Part 2: How Lmod actually evaluates a modulefile
    \item There is a surprising amount to talk about before the
      modulefile gets evaluated.
    \item We will be talking about a user requested load.
    \item A module requested load is slightly different. 
  \end{itemize}
\end{frame}

% page 3
\begin{frame}[fragile]
    \frametitle{Outline (II)}
  \begin{itemize}
    \item \texttt{main()}: How main() parses the command line
    \item \texttt{Load\_Usr()}: Starts working your modules to load
    \item \texttt{l\_usrLoad()}: Splits modules to loads or unloads
    \item \texttt{MName Class}: Maps module name to a filename
    \item \texttt{mcp}: Object to control kind of evaluation
    \item \texttt{MasterControl:load(mA)}: We are loading a modulefile
    \item \texttt{Master:load(mA)}: Where the heavy lifting is done
    \item \texttt{loadModuleFile()}: Where the modulefile is actually evaluated
  \end{itemize}
\end{frame}

% page 4
\begin{frame}[fragile]
    \frametitle{\texttt{main()}: parsing command line}
  \begin{itemize}
    \item main() is located in lmod.in.lua (installed as lmod)
    \item The command is: {\color{blue}\emph{module load foo}}
    \item To parse Lmod uses the 2nd argument w/o a minus: {\color{blue}\emph{load}}
    \item Lmod searches \texttt{lmodCmdA} to map string to command
  \end{itemize}
\end{frame}

% page 5
\begin{frame}{A side note on Lmod Coding Conventions}
  \begin{itemize}
    \item A variable w/ a trailing ``A'' means that an array
    \item A variable w/ a trailing ``T'' means that an table (or dictionary)
    \item A variable w/ a trailing ``Tbl'' means that an table (or dictionary)
    \item A routine with a l\_name is a local function (file scope)
    \item Class Name are in CamelCase
  \end{itemize}
\end{frame}

% page 6
\begin{frame}[fragile]
    \frametitle{lmodCmdA}
 {\tiny
   \begin{semiverbatim}
   local lmodCmdA = \{
      \{cmd = 'add',          min = 2, action = loadTbl     \},
      \{cmd = 'avail',        min = 2, action = availTbl    \},
      \{cmd = 'isloaded',     min = 3, action = isLoadedTbl \},
      \{cmd = 'is_loaded',    min = 4, action = isLoadedTbl \},
      \{cmd = 'is-loaded',    min = 4, action = isLoadedTbl \},
      \{cmd = 'load',         min = 2, action = loadTbl     \},
      ...
   \}
   local loadTbl      = \{ name = "load", cmd = Load_Usr    \}
    \end{semiverbatim}
  \begin{itemize}
    \item So ``load'' matches ``load'' with more than 2 chars
  \end{itemize}
}
\end{frame}

% page 7
\begin{frame}[fragile]
    \frametitle{Going from lmodCmdA to Load\_Usr()}
  \begin{itemize}
    \item Sets cmdT to loadTbl
    \item Sets cmdName to cmdT.name (forces standard name not user
      command name)
    \item cmdT.cmd(unpack(masterTbl.pargs)) $\Rightarrow$ Load\_Usr
  \end{itemize}

\end{frame}

% page 8
\begin{frame}[fragile]
    \frametitle{Calling Load\_Usr()}
  \begin{itemize}
    \item All functions implementing user commands are in src/cmdfunc.lua
  \end{itemize}

 {\tiny
    \begin{semiverbatim}
function Load\_Try(...)
   dbg.start{"Load\_Try(",concatTbl(\{...\},", "),")"}
   local check\_must\_load = false
   local argA            = pack(...)
   l\_usrLoad(argA, check\_must\_load)
   dbg.fini("Load\_Try")
end

function Load\_Usr(...)
   dbg.start{"Load\_Usr(",concatTbl(\{...\},", "),")"}
   local check\_must\_load = true
   local argA            = pack(...)
   l\_usrLoad(argA, check\_must\_load)
   dbg.fini("Load\_Usr")
end
    \end{semiverbatim}
}
\end{frame}

% page 9
\begin{frame}[fragile]
    \frametitle{l\_usrLoad(argA, check\_must\_load)}
  \begin{itemize}
    \item Split argA into loads in lA, unloads in uA (-foo)
    \item Both uA and lA are an array of MName objects.
    \item unload modules in uA 
    \item {\color{blue} \texttt{lA[\#lA+1] = MName:new("load",module\_name)}}
    \item {\color{blue} \texttt{mcp:load\_usr(lA)}}
    \item src: src/cmdfunc.lua
  \end{itemize}
\end{frame}

% page 10
\begin{frame}[fragile]
    \frametitle{MName class: Module Name class}
 {\tiny
    \begin{semiverbatim}
    \end{semiverbatim}
}
  \begin{itemize}
    \item Maps name (``foo'' or ``foo/1.1'') to filename
    \item There are two kinds of searching ``load'' or ``mt''
    \item Load: must search file system. 
    \item mt: filename is in moduletable
    \item Evaluation must be lazy or just-in-time
    \item Software hierarchy means that 
    \item {\color{blue}\texttt{module load gcc mpich}}
    \item mpich might not be in \$MODULEPATH until after gcc is loaded
    \item src: src/MName.lua
  \end{itemize}

\end{frame}

% page 11
\begin{frame}{MName key concepts}
  \begin{itemize}
    \item \textbf{userName}: name on the command line
    \item It might be gcc or gcc/9.3.0
    \item \textbf{sn}: the shortName or a name without a version
    \item \textbf{fullName}: The full name of the module (sn/version)
    \item Examples:
      \begin{enumerate}
        \item gcc/9.3.0 (sn: gcc, N/V)
        \item gcc/x86\_64/9.3.0 (sn: gcc, N/V/V)
        \item compiler/gcc/9.3.0 (sn: compiler/gcc, C/N/V)
        \item compiler/gcc/x86\_64/9.3.0 (sn: compiler/gcc, N/V/V)
      \end{enumerate}
  \end{itemize}
\end{frame}

% page 12
\begin{frame}[fragile]
    \frametitle{mcp and MasterControl class}
  \begin{itemize}
    \item MasterControl class is what controls whether a "load" in a
      modulefile is a load or unload
    \item mcp is a global variable that is built to be in a mode()
      like load, unload, spider, etc.
    \item We talked about this in an earlier presentation.
  \end{itemize}
\end{frame}

% page 13
\begin{frame}[fragile]
    \frametitle{MasterControl:load(mA)}
 {\tiny
    \begin{semiverbatim}
function M.load(self, mA)
   local master = Master:singleton()
   local a      = master:load(mA)

   if (not quiet()) then
      self:registerAdminMsg(mA) -- nag msg registration.
   end
   return a
end
    \end{semiverbatim}
}
  \begin{itemize}
    \item MasterControl functions call Master Functions to do the work.
    \item src: src/MasterControl.lua
  \end{itemize}
\end{frame}

% page 14
\begin{frame}[fragile]
    \frametitle{Master:load(mA)}
 {\tiny
    \begin{semiverbatim}
function M.load(mA)
  for i = 1, #mA do
     repeat
        mname = mA[i]
        sn = mname:sn()  -- shortName
        fn = mname:fn()  -- file name
        -- if blank sn -> pushModule (might have to wait until
        -- compiler or mpi is loaded.
        -- and break (really continue)

        if (mt:have(sn,"active)) then
           -- Block 1: Check for previously loaded module with same sn

        elseif (fn) then
           -- Block 2: Load modulefile

        -- Check for family stack (e.g. compiler, mpi etc)
        if (mcp.processFamilyStack(fullName)) then
            -- Suppose gcc is loaded and it was "replaced" by intel
            -- unload gcc and reload intel
         end
      until true
   end         
     
   -- Reload every module if change in MODULEPATH.
     
   -- load any modules on module stack
end
    \end{semiverbatim}
}
  \begin{itemize}
    \item This is where the heavy lifting is done.
    \item src: src/Master.lua
  \end{itemize}

\end{frame}

% page 15
\begin{frame}[fragile]
    \frametitle{Block 1: Check for previous loaded module w/same sn}
 {\tiny
    \begin{semiverbatim}
 if (mt:have(sn,"active)) then
    -- if disable_same_name_autoswap -> error out
    -- Otherwise: unload previous module
     local mcp_old = mcp
     local mcp     = MCP
     unload_internal{MName:new("mt",sn)}
     mname:reset()  -- force a new lazyEval
     local status = mcp:load_usr\{mname\}
     mcp          = mcp_old
    \end{semiverbatim}
}
  \begin{itemize}
    \item Here we guarantee the right mcp
    \item Unload the old module 
    \item Recursively mcp:load\_usr{mname} 
  \end{itemize}
\end{frame}

% page 16
\begin{frame}[fragile]
    \frametitle{Block 2: Load modulefile}
 {\tiny
    \begin{semiverbatim}
elseif (fn) then
   frameStk:push(mname)
   mt = frameStk:mt()
   mt:add(mname,"pending")
   local status = loadModuleFile\{file = fn, shell = shellNm,
                   mList = mList, reportErr = true\}
   frameStk:pop()
   loaded = true
end
    \end{semiverbatim}
}
\end{frame}

% page 17
\begin{frame}[fragile]
    \frametitle{loadModuleFile(t)}
  \begin{itemize}
    \item This is where Lmod handle either *.lua files or TCL Modulefiles
    \item Once either read in as a block (for *.lua) or converted (TCL
      modulefile $\Rightarrow$ Lua)
    \item src: src/loadModulefile.lua
  \end{itemize}
 {\tiny
    \begin{semiverbatim}
   -- Use the sandbox to evaluate modulefile text.
   if (whole) then
      status, msg = sandbox_run(whole)
   else
      status = nil
      msg    = "Empty or non-existent file"
   end
   -- report any errors
    \end{semiverbatim}
}

\end{frame}

% page 18
\begin{frame}{Next time}
  \begin{itemize}
    \item What is a sandbox and how does it work?
    \item Why I want a sandbox?
    \item Next time handing control to modulefile
  \end{itemize}
\end{frame}



% page 19
\begin{frame}{Conclusions}
  \begin{itemize}
    \item It takes a lot to get to the point where Lmod is evaluating
      your modulefile.
    \item Lmod uses several ``classes'' to manage the loading of a
      module
    \item Plus a couple of Design Patterns such as Singletons  
  \end{itemize}
\end{frame}

% page 20
\begin{frame}{Future Topics}
  \begin{itemize}
    \item Next Meeting: July 5th 9:30 US Central (14:30 UTC)
    \item What happens from the loadModulefile(t).
    \item This is where Lmod hands off control to the user.
  \end{itemize}
\end{frame}

\end{document}
