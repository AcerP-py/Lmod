\documentclass{beamer}

% You can also use a 16:9 aspect ratio:
%\documentclass[aspectratio=169]{beamer}
\usetheme{TACC16}

% It's possible to move the footer to the right:
%\usetheme[rightfooter]{TACC16}

\begin{document}
\title[Lmod]{Lmod Star Meeting}
\author{Robert McLay} 
\date{Sept. 21, 2022}

% page 1
\frame{\titlepage} 

\section{Introduction}

% page 2
\begin{frame}{Introduction}
  \center{\includegraphics[width=.9\textwidth]{Lmod-4color@2x.png}}
  \begin{itemize}
    \item Features and History
    \item Advanced Topics
    \item Future work?
  \end{itemize}
\end{frame}

% page 3
\begin{frame}{Features}
  \begin{itemize}
    \item Current version is Lmod 8.7.13
    \item Reads for TCL and Lua modulefiles
    \item One name rule.
    \item Support Software Hierarchy (but not required!)
    \item Spider Cache: fast \texttt{\color{blue} \$ module avail}
    \item Properties (gpu, mic)
    \item family(``compiler'') family(``mpi'') support
    \item Optional Tracking: What modules are loaded?
    \item Many other features: ml, collections, hooks,
      extended default, nag ...
    \item Google analytics report about 1000 user read docs.
  \end{itemize}
\end{frame}

% page 4
\begin{frame}{\texttt{Recent Feature of Lmod 8+}}
  \begin{itemize}
    \item The TCL interpreter is now (optionally) embedded with Lmod.
    \item \texttt{depends\_on()}
    \item New Function: \texttt{extensions("numpy/1.16.4 scipy/1.4")}
    \item Checking your module tree 8.4.3+
  \end{itemize}
\end{frame}

% page 5
\begin{frame}{\texttt{depends\_on()}}
  \begin{itemize}
    \item Modules X and Y depends on Module A
    \item ml purge; ml X; ml unload X;      $\Rightarrow$ unload A
    \item ml purge; ml X Y; ml unload X;    $\Rightarrow$ keep A
    \item ml purge; ml X Y; ml unload X Y ; $\Rightarrow$ unload A
    \item ml purge; ml A X Y; ml unload X Y ; $\Rightarrow$ keep A
  \end{itemize}
\end{frame}

% page 6
\begin{frame}{extensions() function}
  \begin{itemize}
    \item extensions(): Tells users that a module has extensions
    \item E.G: python has numpy and scipy
    \item \texttt{extensions("numpy/1.16.4, scipy/1.4")}
  \end{itemize}
\end{frame}

% page 7
\begin{frame}{extensions() function (II)}
  \begin{itemize}
    \item Users can use spider to find extensions.
    \item Users can use avail to list extensions base name
  \end{itemize}
\end{frame}

% page 8
\begin{frame}{Checking your module tree 8.4.3+}
  \begin{itemize}
    \item New command added: \texttt{\$LMOD\_DIR/check\_module\_tree\_syntax}
    \item Reports syntax errors across the entire \texttt{\$MODULEPATH}
    \item Report which modules have multiple marked defaults sets
    \item Precedent order: default symlink, .modulerc.lua, .modulerc, .version
    \item Does not check SYSTEM MODULERCFILE for defaults.
  \end{itemize}
\end{frame}

% page 9
\begin{frame}{Lmod 8.6+ Features}
  \begin{itemize}
    \item New Features now reported:
      https://lmod.readthedocs.io/en/latest/025\_new.html
    \item \texttt{module overview}
    \item \texttt{module -d avail}
    \item New config file: /etc/lmod/lmod\_config.lua
    \item LMOD\_QUARANTINE\_VARS
    \item updates to sh\_to\_modulefile
    \item source\_sh(): source a shell script inside a modulefile
  \end{itemize}
\end{frame}

% page 9a
\begin{frame}[fragile]
  \frametitle{module overview}
    {\tiny
\begin{semiverbatim}
% module overview
------------------ /opt/apps/modulefiles/Core -----------------
StdEnv    (1)   hashrf    (2)   papi        (2)   xalt     (1)
ddt       (1)   intel     (2)   singularity (2)
git       (1)   noweb     (1)   valgrind    (1)

--------------- /opt/apps/lmod/lmod/modulefiles/Core ----------
lmod (1)   settarg (1)    
\end{semiverbatim}
    }
\end{frame}

% page 10
\begin{frame}{\$LMOD\_QUARANTINE\_VARS}
  \begin{itemize}
    \item A module at TACC turn-off  \$LMOD\_PAGER
    \item This \!\@\#\%\& module made me mad.
    \item Tmod has a new feature kinda like this.
    \item \$LMOD\_QUARANTINE\_VARS was invented.
  \end{itemize}
\end{frame}

% page 11
\begin{frame}{\$LMOD\_QUARANTINE\_VARS (II)}
  \begin{itemize}
    \item export LMOD\_QUARANTINE\_VARS=LMOD\_PAGER:LMOD\_REDIRECT
    \item This means that a module can't change those variables.
    \item This only works with regular env. vars.
    \item You can't quarantine PATH like variables.
    \item A user sets this variable in their $\sim$/.bashrc or similar
      file.
    \item This obviously won't work for modules loaded during the
      processing of /etc/profile.d/*.sh files
    \item Use https://github.com/TACC/ShellStartupDebug support users.
  \end{itemize}
\end{frame}

% page 12
\begin{frame}[fragile]
  \frametitle{/etc/lmod/lmod\_config.lua configuration file}
  \begin{itemize}
    \item This file is evaluated during Lmod startup. 
    \item This location is the default during configuration.
    \item A site can change this location at configuration.
  \end{itemize}
    {\small
\begin{semiverbatim}
-- Example /etc/lmod/lmod\_config.lua
require("strict")
local cosmic = require("Cosmic"):singleton()

cosmic:assign("LMOD\_SITE\_NAME", "XYZZY")
local function foo()
  ...
end
sandbox\_registration \{ foo = foo \}
\end{semiverbatim}
}
\end{frame}

% page 13
\begin{frame}{Sourcing shell scripts inside a modulefile w/ source\_sh()}
  \begin{itemize}
    \item This was first implemented in Tmod 4.7
    \item Xavier told me that he did this during Covid Lockdown in France.
    \item Lmod 8.6 re-implements this feature in a similar way.
    \item It knows about env. vars and shell functions and aliases.
  \end{itemize}
\end{frame}

% page 14
\begin{frame}{source\_sh() Implementation}
  \begin{itemize}
    \item It captures the env. vars/functions/alias before and after
      the running the shell script.
    \item It computes the difference and saves it into the ModuleTable
      in env.
    \item It can be safely unloaded, shown.
    \item script path and arguments must not change between load and unload.
    \item \texttt{module refresh} works
    \item Obvious points:
      \begin{itemize}
        \item It is better to use sh\_to\_modulefile and convert once.
        \item But sh\_to\_modulefile is not dynamic (e.g. \$HOME)
        \item Can't have run the script in the user environment before
          loading the script.
      \end{itemize}
  \end{itemize}
\end{frame}

% page 15
\begin{frame}[fragile]
  \frametitle{ml --mt}
    {\tiny
\begin{semiverbatim}
\_ModuleTable\_ = \{
  MTversion = 3,
  mT = \{
    wrapperSh = \{
      fn = "/home/user/w/lmod/rt/sh_to_modulefile/mf/wrapperSh/1.0.lua",
      fullName = "wrapperSh/1.0",
      loadOrder = 1,
      mcmdT = {
        ["/home/user/w/lmod/rt/sh_to_modulefile/second.sh arg1"] = \{
          "setenv(\\"SECOND\\",\\"FOO_BAR\\")",
        \},
        ["/home/user/w/lmod/rt/sh_to_modulefile/tstScript.sh"] = \{
          "setenv(\\"MY_NAME\",\\"tstScript.sh\\")",
          "prepend_path(\\"PATH\\",\\"/home/user/w/lmod/rt/sh_to_modulefile/bin\\")",
          "set_alias(\\"fooAlias\\",\\"foobin -q -l\\")"
          , [[set_shell_function("banner"," \\
    local str=\\"$1\\";\\
    local RED='\\27[1;31m';\\
    local NONE='\\27[0m';\\
    echo \\"${RED}${str}${NONE}\\"\\
")]], 
        \},
      \},
    \},
  \},
}
\end{semiverbatim}
    }
\end{frame}

% page 16
\begin{frame}[fragile]
  \frametitle{Interesting Bug in Bash and shell functions}
    {\tiny
\begin{semiverbatim}
    set_shell_function("_some_spack_func" "\textbackslash
       local ARG1=\$1\textbackslash
       if [[ \$ARG1 == {\color{red}[a-z]*} ]]; then\textbackslash
         echo ...\textbackslash
       fi\textbackslash
    ","")
\end{semiverbatim}
    }
  \begin{itemize}
    \item This works fine in zsh but not bash
    \item All bash versions expand [a-z]\** to files in current directory
    \item I see no way to fix this
    \item This fails because of the eval step
    \item I am going to change the default to ignore {\color{blue}\_func...() \{ \}}
    \item This fails for Tmod 4+ as well.
    \item Bash always expands {\color{blue}A='[a-z]\**'}
  \end{itemize}
\end{frame}

% page 17
\begin{frame}{Lmod Monthly Zoom Mtg}
  \begin{itemize}
    \item Previous Topics
      \begin{enumerate}
        \item Lmod Hooks Discussion
        \item Debugging Modulefiles
        \item Lmod 8.6 new features
        \item Settarg and integrating Lmod with build system
        \item How Module collections works 
      \end{enumerate}
    \item Future Topics: 
      \begin{enumerate}
        \item How to get current module info into hooks
        \item How Lmod testing works
      \end{enumerate}
    \item Zoom Mtg Usually 1st Tuesday at 15:30 UTC (9:30 US Central)
    \item See Mailing list or https://github.com/TACC/Lmod/wiki for
      details
    \item Next Meeting Oct 4th at 15:30 UTC (9:30 US Central)
  \end{itemize}
\end{frame}

% page 18
\begin{frame}{Future Work}
  \begin{itemize}
    \item Lmod can optionally track usage.
    \item Future: Make it easier to not remember loads after 1 year.
    \item A monthly discussion group? (YES!!)
  \end{itemize}
\end{frame}

% page 19
\begin{frame}{Conclusions: Lmod 8+}
  \center{\includegraphics[width=.9\textwidth]{Lmod-4color@2x.png}}
  \begin{itemize}
    \item Latest version: https://github.com:TACC/lmod.git
    \item Stable version: http://lmod.sf.net
    \item Documentation:  http://lmod.readthedocs.org
  \end{itemize}
\end{frame}

\end{document}
